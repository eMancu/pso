\documentclass[10pt, a4paper]{article}

\usepackage[paper=a4paper, left=1.5cm, right=1.5cm, bottom=1.5cm, top=3.5cm]{geometry}
\usepackage[utf8]{inputenc}
\usepackage[spanish]{babel}
\usepackage{framed}
\usepackage{endnotes}
\usepackage{graphicx}
\usepackage{multicol}
\usepackage{color}

\newcommand{\notaClase}[1]{\subparagraph{NC.: } {\color{blue}\footnotesize  #1}}
\renewcommand\notesname{Notas Finales}

%Datos para la caratula
\title{Exokernel \\ \small{Apuntes de presentación}}

\author{E. Mancuso - A. Mataloni - M. Miguel}

\date{16 de Junio de 2011}

\begin{document}

\maketitle

\section{Introducción}
El concepto de Exokernel es una propuesta de una nueva filosofía en la confección de kernels. Esta propuesta está motivada por las siguientes observaciones:

Los kernels de hoy en día...
\begin{itemize}
 \item ... son poco flexibles.
 \item ... día deben dar soporte a una gran variedad de tipo de aplicaciones.
 \item ... pierden eficiencia en la generalidad y el grosor de trabajo que deben hacer.
\end{itemize}



\end{document}